\documentclass[english]{article}
\usepackage[T1]{fontenc}
\usepackage[latin9]{inputenc}
\usepackage{xcolor}
\usepackage{amssymb}
\usepackage{babel}
\begin{document}

\subsection*{�lgebra Lineal Final Te�rico}
\begin{itemize}
\item Conjuntos linealmente independientes:\\
Un conjunto es linealmente independiente si $a_{1}v_{1}+a_{2}v_{2}+\cdots+a_{n}v_{n}=0\Rightarrow a_{1}=\cdots=a_{n}=0$.
\item Suma directa\\
\\
La suma de dos subespacios es directa si y s�lo si la intersecci�n
de los subespacios es el vector nulo.\\
Sean $U_{1}$ y $U_{2}$ subespacios de V, luego las siguientes proposiciones
son equivalentes:
\begin{itemize}
\item $V=U_{1}\bigoplus U_{2}$
\item $V=U_{1}+U_{2}$ y $0=u_{1}+u_{2}\Rightarrow u_{1}=u_{2}=0$
\item $V=U_{1}+U_{2}$ y $U_{1}\cap U_{2}=\{0\}$
\end{itemize}
\item Demostrar que S es LI si y s�lo si el generado de S es suma directa
de los espacios generados de los vectores de S.\\
Sea $S=\{s_{1},\cdots,s_{n}\}$
\begin{itemize}
\item $\Rightarrow$ Supongo que $S$ es un conjunto linealmente independiente
entonces existen vectores $s_{1},\cdots,s_{n}\in S$ distintos y escalares
$a_{1}+\cdots+a_{n}$\textcolor{purple}{{} }tales que $a_{1}s_{1}+\cdots+a_{n}s_{n}=0\Rightarrow a_{1}=\cdots=a_{n}=0$.\\
Luego se cumple que $s_{i}+s_{j}=0\Rightarrow s_{i}=s_{j}=0,con\ i\not=j$,
por lo tanto se cumple que $\langle S\rangle=\sum_{s\in S}\alpha_{s}s$.
\item $\Leftarrow$ Supongo que S es suma directa de los espacios generados
de los vectores de S.\\
Entonces se cumple que $s_{1}+\cdots+s_{n}=0\Rightarrow s_{1}=\cdots=s_{n}=0$
por lo tanto S es un conjunto linealmente independiente.
\end{itemize}
\item Lema de Schur\\
\\
Sea V un espacio vectorial sobre $\mathbb{K}$ y sea $S=\{U_{1},\cdots,U_{n}\}$
un conjunto de operadores de $V$. Si $V$ es un S-espacio simple
y $T:V\rightarrow V$ una transformaci�n lineal tal que $TU_{i}=U_{i}T$
para toda $U_{i}\in S$, entonces $T$ es invertible o $T$ es la
aplicaci�n nula.\\
\textbf{Demostraci�n}\\
\\
Supongamos $T\not=0$. Por la proposici�n 1 sabemos que $img(T)$
y $nul(T)$ son subespacios invariantes. Adem�s como V es un S-espacio
simple sus �nicos subespacios invariantes son V y $\{0\}$ por lo
que $img(T)=V$ y $nul(T)=\{0\}$. Esto implica que $T$ es sobreyectiva
e inyectiva, luego es invertible.
\item Proceso de Gram-Schmidt
\item Teoremas de Pit�goras
\item Producto Interno
\item Teorema de Jordan
\item Definici�n de matrices semejantes
\item Definici�n de polinomio caracter�stico
\item Definici�n de autovalor, autovector
\item Condici�n necesaria y suficiente para diagonalizar una matriz.
\item Probar que los autovalores de dos matrices semajantes son iguales.
\item A inversible$\Rightarrow N(A)={0}$, las columnas son linealmente
independientes, el determinante distinto a 0.
\item Teorema de descomposici�n ortogonal
\item Sea V un espacio vectorial, $T:V\rightarrow V$. Definir cuando un
espacio V es c�clico (enunciar todo lo necesario para dar la definici�n).
\item Definir base de Jordan y demostrar que es base.
\end{itemize}

\end{document}
