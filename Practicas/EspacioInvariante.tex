\documentclass{article}
\usepackage[utf8]{inputenc}
\usepackage[margin=2cm]{geometry}
\usepackage{mathtools}
\usepackage{amsfonts}
\usepackage{enumerate}
\usepackage{cancel}
\usepackage{tgadventor}
{\fontfamily{qag}\selectfont 
\title{Espacios invariantes}
\begin{document}

\maketitle
\section{Rinconmatemático}

$
\begin{bmatrix}
  1 & 0 & 0 \\
  0 & 1 & 0 \\
  3 & -5 & 2 \\
\end{bmatrix} \\
\left [
\begin{array}{ccc}
  \alpha_1 \\
  \alpha_2 \\
  \alpha_3
\end{array} 
\right ] = 
\left [
  \begin{array}{ccc}
    \alpha_1 \\
    \alpha_2 \\
    3\alpha_1-5\alpha_2+2\alpha_3
  \end{array}
\right ]
$ \\ \\
Sustituyendo este transformado en la ecuación de $V_1$ obtenemos: \\ \\
$3\alpha_1 - 5\alpha_2 + 3\alpha_1 - 5 \alpha_2 + 2\alpha_3 = 6\alpha_1 - 10\alpha_2 + 
2\alpha_3 = 0 \Leftrightarrow 3\alpha_1 - 5\alpha_2 + \alpha_3 = 0$ \\ \\
Es decir, el transformado de todo vector de $V_1$ está en $V_1$, en consecuencia ya
tenemos un subespacio invariante de dimensión 2.
}



\end{document}
